\documentclass[12pt,twoside]{article}
\usepackage{amsmath, amssymb}
\usepackage{amsmath}
\usepackage[active]{srcltx}
\usepackage{amssymb}
\usepackage{amscd}
\usepackage{makeidx}
\usepackage{amsthm}
\usepackage{algpseudocode}
\usepackage{algorithm}
\usepackage{array}
\newcolumntype{L}{>{\centering\arraybackslash}m{3cm}}
\renewcommand{\baselinestretch}{1}
\setcounter{page}{1}
\setlength{\textheight}{21.6cm}
\setlength{\textwidth}{14cm}
\setlength{\oddsidemargin}{1cm}
\setlength{\evensidemargin}{1cm}
\pagestyle{myheadings}
\thispagestyle{empty}
\markboth{\small{Josu\'e David Hern\'andez Ram\'irez}}{\small{.}}
\date{}
\begin{document}
\centerline{\bf Ingeniería de Software, Sem: 2021-1, 3CV3, Tarea 3, 4/12/2020}
\centerline{}
\centerline{}
\begin{center}
\Large{\textsc{Tarea 3: Plan de alcance}}
\end{center}
\centerline{}
\centerline{\bf {Josu\'e David Hern\'andez Ram\'irez.}}
\centerline{}
\centerline{Escuela Superior de C\'omputo}
\centerline{Instituto Polit\'ecnico Nacional, Ciudad de M\'exico}
\centerline{$jhernandezr1605@alumno.ipn.mx$}
\newtheorem{Theorem}{\quad Theorem}[section]
\newtheorem{Definition}[Theorem]{\quad Definition}
\newtheorem{Corollary}[Theorem]{\quad Corollary}
\newtheorem{Lemma}[Theorem]{\quad Lemma}
\newtheorem{Example}[Theorem]{\quad Example}
\bigskip

\section{Statement of Work}
Es un proyecto para ayudar al gobierno de la ciudad de México a recibir reportes de los
baches que hay en la ciudad.

\section{Identificación de stakeholder}
\subsection{Stakeholders internos}
\begin{itemize}
    \item \textbf{Analista: }Es el encargo de analizar las necesidades del cliente
        y traducirlas para que los desarrolladores entiendan que es lo que
        requiere el cliente y realizarlo.
    \item \textbf{Desarrollador Frontend: }Es el encargado del desarrollo de
        interfaz de usuario y la experencia dentro del sistema.
    \item \textbf{Desarrollador Backend: }Es el encargado de desarrollar
        la lógica de la aplicación y darle mantenimiento a los servicios que
        ofrece esta.
    \item \textbf{Project Manager: }Es la persona encargada de dirigir el proyecto
        y cumplir las necesidades y requerimientos del cliente y de los usuarios.
    \item \textbf{Cliente :}Es la persona o empresa a la que vamos a cubrir sus 
        necesidades mediante el desarrollo de un sistema.
\end{itemize}

\subsection{Stakeholders externos}
\begin{itemize}
    \item \textbf{Jefa de Gobierno: }es el titular del poder ejecutivo de la entidad.
        Sus funciones están descritas en el artículo 32, apartado C de la Constitución
        de la Ciudad de México. La dirección en donde se puede encontrar es en Plaza 
        de la Constitución 2, Centro Histórico de la Cdad. de México, 06000 
        Cuauhtémoc, CDMX.
    \item \textbf{SEMOVI: }Se encarga de regular, programar, orientar, organizar, 
        controlar, aprobar y, en su caso, modificar, la presentación de los servicios 
        público, mercantil y privado de transporte de pasajeros y de carga en la 
        Ciudad de México, conforme a lo establecido en la Ley y demás disposiciones 
        jurídicas y administrativas aplicables; así como también, a las necesidades de 
        movilidad de la Ciudad, procurando la preservación del medio ambiente y la 
        seguridad de los usuarios del sistema de movilidad. Se encuentra en Avenida 
        Álvaro Obregón 269 Colonia Roma Norte, Alcaldía Cuauhtémoc C.P. 06700, Ciudad 
        de México.
    \item \textbf{Secretaría de Obras y Servicios: }En esta dependencia del Gobierno 
        de la Ciudad de México se encarga de establecer la normatividad y las 
        especificaciones aplicables a la obra pública, concesionada y los servicios 
        urbanos; planean, proyectan, construyen, mantienen y operan con un 
        enfoque integral y una visión metropolitana acorde al propósito de garantizar 
        el desarrollo sustentable. La Secretaría de Obras y Servicios, propone nuevos 
        estándares de construcción en la obra pública, integra elementos de sustentabilidad, 
        accesibilidad, elementos modernos que cumplan con las necesidades de una Capital 
        en crecimiento y desarrollo continuo. Integramos proyectos ejecutivos, construimos 
        obras cuya planeación, programación y operación corresponde a otras dependencias 
        del Gobierno de la Ciudad de México, como instalaciones educativas, hospitalarias, 
        deportivas, culturales, centros de atención social, entre otras. Para lograr una 
        adecuada coordinación, las distintas dependencias con las que trabajamos se da 
        un seguimiento programático presupuestal en materia de obras en el Comité de Obras 
        de la Capital. Se encuentran en Calle Plaza de la Constitución 1, Colonia Centro 
        (Área 1), Alcaldía Cuauhtémoc, C.P. 06000, Ciudad de México.
    \item \textbf{Ciudadano: }Son los usuarios finales, a los cuales la propuesta del 
        sistema va dirigido para ayudar a mejorar la ciudad mediante sus reportes.
\end{itemize}

\section{Lista de necesidades}
La siguiente lista de necesidades tiene establecido el nivel de prioridad de 1 a 5,
siendo 1 alto y 5 bajo. \\

\begin{table}[!hbt]
        \begin{tabular}{|l|l|l|l|}
        \hline
        \multicolumn{1}{|c|}{Id} & \multicolumn{1}{c|}{Nombre} & \multicolumn{1}{c|}{Descripción} & \multicolumn{1}{c|}{Prioridad} \\ \hline
        \multicolumn{1}{|r|}{N01} & Reportar baches & \begin{tabular}[c]{@{}l@{}}Es necesario implementar una \\ solución para el reporte de estos\end{tabular} & 1 \\ \hline
        \multicolumn{1}{|r|}{N02} & \begin{tabular}[c]{@{}l@{}}Diferentes tipos de\\ usuario\end{tabular} & \begin{tabular}[c]{@{}l@{}}Debe de existir una calsificación \\ de usuarios\end{tabular} & 2 \\ \hline
        \multicolumn{1}{|r|}{N03} & Estatus del bache & \begin{tabular}[c]{@{}l@{}}Se debe de mostrar el estatus del \\ bache a los usuarios\end{tabular} & 3 \\ \hline
        \multicolumn{1}{|r|}{N04} & Ubicación del bache & \begin{tabular}[c]{@{}l@{}}Se debe de tener la ubicación \\ del bache\end{tabular} & 1 \\ \hline
        \multicolumn{1}{|r|}{N05} & \begin{tabular}[c]{@{}l@{}}Reporte para secretaría \\ de obras\end{tabular} & \begin{tabular}[c]{@{}l@{}}El sistema debe crear un reporte \\ para que se repare el desperfecto\end{tabular} & 2 \\ \hline
        \multicolumn{1}{|r|}{N06} & Seguridad de datos & \begin{tabular}[c]{@{}l@{}}Los datos de los usuarios deben \\ de estar seguros\end{tabular} & 2 \\ \hline
        N07 & \begin{tabular}[c]{@{}l@{}}Licencia de \\ Google Maps\end{tabular} & \begin{tabular}[c]{@{}l@{}}Tener el acceso a las API's de \\ google maps para acceder a los \\ servicios de localización\end{tabular} & 2 \\ \hline
        N08 & Acceso a internet & \begin{tabular}[c]{@{}l@{}}Es necesario tener comunicación \\ con el equipo y el cliente, para ello \\ es necesario contar con este \\ servicio\end{tabular} & 1 \\ \hline
        N09 & Licencia de MySQL & \begin{tabular}[c]{@{}l@{}}Es necesario tener todos los \\ servicios disponibles de la \\ base de datos.\end{tabular} & 3 \\ \hline
        N10 & Documentación & \begin{tabular}[c]{@{}l@{}}Para mantener el sistema y \\ mejorarlo con el tiempo, es \\ necesario saber que se hizo \\ para repararlo o mejorarlo\end{tabular} & 4 \\ \hline
        N11 & Equipo de cómputo & \begin{tabular}[c]{@{}l@{}}Es la herramienta principal para \\ desarrollar el sistema\end{tabular} & 1 \\ \hline
        N12 & Interfaz amigable & \begin{tabular}[c]{@{}l@{}}Es necesario que el sistema \\ sea de fácil entendimiento \\ para los usuarios.\end{tabular} & 2 \\ \hline
        \end{tabular}
        \end{table}

\section{Objetivos específicos}
\begin{itemize}
    \item \textbf{Objetivo 1}
\end{itemize}

\section{Objetivo general del proyecto}


\section{Entregables}


\section{Lista de requerimientos funcionales y no funcionales}


\end{document}

