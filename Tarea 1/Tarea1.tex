\documentclass[12pt,twoside]{article}
\usepackage{amsmath, amssymb}
\usepackage{amsmath}
\usepackage[active]{srcltx}
\usepackage{amssymb}
\usepackage{amscd}
\usepackage{makeidx}
\usepackage{amsthm}
\usepackage{algpseudocode}
\usepackage{algorithm}
\renewcommand{\baselinestretch}{1}
\setcounter{page}{1}
\setlength{\textheight}{21.6cm}
\setlength{\textwidth}{14cm}
\setlength{\oddsidemargin}{1cm}
\setlength{\evensidemargin}{1cm}
\pagestyle{myheadings}
\thispagestyle{empty}
\markboth{\small{Josu\'e David Hern\'andez Ram\'irez}}{\small{.}}
\date{}
\begin{document}
\centerline{\bf Ingeniería de Software, Sem: 2021-1, 3CV3, Tarea 1, 30/09/2020}
\centerline{}
\centerline{}
\begin{center}
\Large{\textsc{Tarea 1: Pertinencia del temario}}
\end{center}
\centerline{}
\centerline{\bf {Josu\'e David Hern\'andez Ram\'irez.}}
\centerline{}
\centerline{Escuela Superior de C\'omputo}
\centerline{Instituto Polit\'ecnico Nacional, M\'exico}
\centerline{$jhernandezr1605@alumno.ipn.mx$}
\newtheorem{Theorem}{\quad Theorem}[section]
\newtheorem{Definition}[Theorem]{\quad Definition}
\newtheorem{Corollary}[Theorem]{\quad Corollary}
\newtheorem{Lemma}[Theorem]{\quad Lemma}
\newtheorem{Example}[Theorem]{\quad Example}
\bigskip
% \textbf{Resumen:} Redactar de manera breve y concisa de que trata el trabajo presentado. Un
% s\'olo p\'arrafo.
% {\bf Palabras Clave:} Colocar de 3 a 5 palabras clave.
\section{Temas}
\subsection{Ingenier\'ia de Software}
\subsubsection{Conceptos b\'asicos de ingenier\'ia de software.}
\textbf{¿Qu\'e es software?}\\
Programas de c\'omputo y su documentaci\'on asociada: requerimientos, modelos de diseño y manuales de usuario,
puede ser creado desarrollando nuevos programas, configurando sistemas de software genérico o reutilizando
software existente.\\
\textbf{¿Qu\'e es la ingenier\'ia de software?}\\
Una disciplina de la Ingeniería que concierne a todos los aspectos de la producción de software, utilizando las
herramientas y técnicas apropiadas para resolver el problema planteado, de acuerdo a las restricciones de desarrollo y
a los recursos disponibles.\\
\textbf{¿Qu\'e es un proceso de software?}\\
Un conjunto estructurado de actividades cuya meta es el desarrollo o evolución de un software
Algunas actividades genéricas en todos los procesos de software son:
\begin{enumerate}
    \item Especificación: qué debe hacer el software y cuáles son sus especificaciones de desarrollo.
    \item Desarrollo: producción del sistema de software Validación, verificar que el software cumple con lo solicitado por el
    cliente.
    \item Evolución: cambiar/adaptar el software a las nuevas demandas.
\end{enumerate}
\textbf{¿Qué es un modelo de proceso de software?}\\
Representación formal y simplificada de un proceso de software, presentada desde una perspectiva específica.\\
Modelos Genéricos:
\begin{enumerate}
    \item Cascada, separar en distintas fases de especificación y desarrollo.
    \item Desarrollo Iterativo, la especificación, desarrollo y validación están interrelacionados.
    \item Prototipo, un modelo sirve de prototipo para la construcción del sistema final.
    \item Basado en componentes, asume que partes del sistema ya existen y se enfoca a su integración-
\end{enumerate}
\textbf{¿Cuáles son los costos de la ingeniería de software?}
El costo total de un software esta dividido aproximadamente de la siguiente forma:
\begin{itemize}
    \item 60\% costos de desarrollo
    \item 40\% costos de pruebas
    \item Los costos dependen del tipo de sistema que se desarrolla y de los requerimientos del mismo tales como
    desempeño y confiabilidad, la distribución de los costos depende del modelo de desarrollo empleado. 
\end{itemize}
\textbf{¿Qué es CASE?}
CASE es Computer-Aided Software Engineering son programas que son usados para dar
soporte automatizado a las actividades del proceso de software como:
\begin{itemize}
    \item Las herramientas CASE son comúnmente usadas para dar soporte a los métodos de
    software
    \item Módulos de análisis que verifican que las reglas del método se cumplan
    \item Generadores de reportes que facilitan la creación de la documentación del sistema
    \item Generadores de código a partir del modelo del sistema
\end{itemize}
\subsubsection{Atributos y características del software}

\subsection{Proceso de gesti\'on de proyecto}
\subsection{Metodolog\'ias}
\subsection{Calidad y normas de calidad}
\subsection{Modelos de madurez}
\subsection{Temas selectos}

\subsection{}
\end{document}