\documentclass[12pt,twoside]{article}
\usepackage{amsmath, amssymb}
\usepackage{amsmath}
\usepackage[active]{srcltx}
\usepackage{amssymb}
\usepackage{amscd}
\usepackage{makeidx}
\usepackage{amsthm}
\usepackage{algpseudocode}
\usepackage{algorithm}
\renewcommand{\baselinestretch}{1}
\setcounter{page}{1}
\setlength{\textheight}{21.6cm}
\setlength{\textwidth}{14cm}
\setlength{\oddsidemargin}{1cm}
\setlength{\evensidemargin}{1cm}
\pagestyle{myheadings}
\thispagestyle{empty}
\markboth{\small{Josu\'e David Hern\'andez Ram\'irez}}{\small{.}}
\date{}
\begin{document}
\centerline{\bf Ingeniería de Software, Sem: 2021-1, 3CV3, Tarea 1, 30/09/2020}
\centerline{}
\centerline{}
\begin{center}
\Large{\textsc{Tarea 1: Pertinencia del temario}}
\end{center}
\centerline{}
\centerline{\bf {Josu\'e David Hern\'andez Ram\'irez.}}
\centerline{}
\centerline{Escuela Superior de C\'omputo}
\centerline{Instituto Polit\'ecnico Nacional, M\'exico}
\centerline{$jhernandezr1605@alumno.ipn.mx$}
\newtheorem{Theorem}{\quad Theorem}[section]
\newtheorem{Definition}[Theorem]{\quad Definition}
\newtheorem{Corollary}[Theorem]{\quad Corollary}
\newtheorem{Lemma}[Theorem]{\quad Lemma}
\newtheorem{Example}[Theorem]{\quad Example}
\bigskip
% \textbf{Resumen:} Redactar de manera breve y concisa de que trata el trabajo presentado. Un
% s\'olo p\'arrafo.
% {\bf Palabras Clave:} Colocar de 3 a 5 palabras clave.
\section{Temas}
\subsection{Ingenier\'ia de Software}
\subsubsection{Conceptos b\'asicos de ingenier\'ia de software.}
\textbf{¿Qu\'e es software?}\\
Programas de c\'omputo y su documentaci\'on asociada: requerimientos, modelos de diseño y 
manuales de usuario,puede ser creado desarrollando nuevos programas, configurando 
sistemas de software genérico o reutilizando software existente.\\
\textbf{¿Qu\'e es la ingenier\'ia de software?}\\
Una disciplina de la Ingeniería que concierne a todos los aspectos de la producción de 
software, utilizando las herramientas y técnicas apropiadas para resolver el problema 
planteado, de acuerdo a las restricciones de desarrollo y a los recursos disponibles.\\
\textbf{¿Qu\'e es un proceso de software?}\\
Un conjunto estructurado de actividades cuya meta es el desarrollo o evolución de un 
software.\\
Algunas actividades genéricas en todos los procesos de software son:
\begin{enumerate}
    \item Especificación: qué debe hacer el software y cuáles son sus especificaciones 
    de desarrollo.
    \item Desarrollo: producción del sistema de software Validación, verificar que el 
    software cumple con lo solicitado por el cliente.
    \item Evolución: cambiar/adaptar el software a las nuevas demandas.
\end{enumerate}
\textbf{¿Qué es un modelo de proceso de software?}\\
Representación formal y simplificada de un proceso de software, presentada desde una 
perspectiva específica.\\
Modelos Genéricos:
\begin{enumerate}
    \item Cascada, separar en distintas fases de especificación y desarrollo.
    \item Desarrollo Iterativo, la especificación, desarrollo y validación están 
    interrelacionados.
    \item Prototipo, un modelo sirve de prototipo para la construcción del sistema final.
    \item Basado en componentes, asume que partes del sistema ya existen y se enfoca a su 
    integración.
\end{enumerate}
\textbf{¿Cuáles son los costos de la ingeniería de software?}
El costo total de un software esta dividido aproximadamente de la siguiente forma:
\begin{itemize}
    \item 60\% costos de desarrollo
    \item 40\% costos de pruebas
    \item Los costos dependen del tipo de sistema que se desarrolla y de los 
    requerimientos del mismo tales como desempeño y confiabilidad, la distribución de 
    los costos depende del modelo de desarrollo empleado. 
\end{itemize}
\textbf{¿Qué es CASE?}
CASE es Computer-Aided Software Engineering son programas que son usados para dar
soporte automatizado a las actividades del proceso de software como:
\begin{itemize}
    \item Las herramientas CASE son comúnmente usadas para dar soporte a los métodos de
    software.
    \item Módulos de análisis que verifican que las reglas del método se cumplan.
    \item Generadores de reportes que facilitan la creación de la documentación del sistema
    \item Generadores de código a partir del modelo del sistema.
\end{itemize}
\subsubsection{Atributos y características del software}
\textbf{Características}
Para poder comprender lo que es el software (y consecuentemente la ingeniería del 
software), es importante examinar las características del software que lo diferencian de 
otras cosas que los humanos pueden construir.
\begin{enumerate}
    \item \textbf{El software se desarrolla, no se fabrica en un sentido clásico}.
    \item \textbf{El software no se estropea}.
    \item \textbf{Aunque la industria tiende a ensamblar componentes, la mayoría del software 
    se construye a medida}.
\end{enumerate}
\textbf{Atributos}
El software debe proveer la funcionalidad y desempeño requeridos por el usuario y
debe ser mantenible, confiable, eficiente y aceptable.
\begin{itemize}
    \item \textbf{Mantenible:} El software debe poder evolucionar.
    \item \textbf{Confiable:} No debe causar daños económicos o físicos.
    \item \textbf{Eficiente:} No desperdiciar recursos del sistema.
    \item \textbf{Aceptable:} Los usuarios deben de aceptarlo.
    \item Debe ser entendible, utilizable y compatible con otros sistemas.
\end{itemize}
\subsubsection{Importancia y aplicación del software}
\textbf{Importancia del Software}\\

Cada software desarrolla funciones específicas dentro de una diversa gama de aplicaciones, 
y sin duda alguna uno de los programas que mayor utilidad representa dentro de una empresa, 
son los denominados Sistemas de Soporte a la Decisión (DSS).

De esta manera, la toma de decisiones se convierte en una variable crítica de éxito dentro 
de las empresas, y es aquí donde radica la importancia de un DSS. \\ \\
\textbf{Aplicaciones del Software}\\

El software puede aplicarse en cualquier situación en la que se haya definido previamente 
un conjunto específico de pasos procedimentales.

El contenido y el determinismo de la información son factores importantes a considerar 
para determinar la naturaleza de una aplicación de software. El contenido se refiere al 
significado y a la forma de la información de entrada y salida. El determinismo de la 
información se refiere a la predictibilidad del orden y del tiempo de llegada de los datos.

Las siguientes áreas del software indican la amplitud de las aplicaciones potenciales:
\begin{itemize}
    \item \textbf{ Software de sistemas:} es un conjunto de programas que han sido escritos 
    para servir a otros programas.\\
    Algunos programas de sistemas (por ejemplo: compiladores, editores y utilidades de gestión 
    de archivos) procesan estructuras de información complejas pero determinadas.
    \item \textbf{ Software de tiempo real:}  coordina, analiza, controla sucesos del mundo 
    real conforme ocurren, se denomina de tiempo real.
    \item \textbf{ Software de gestión:} Las aplicaciones en esta área re estructuran los datos 
    existentes para facilitar las operaciones comerciales o gestionar la toma de decisiones. \\
    Además de las tareas convencionales de procesamientos de datos, las aplicaciones de software 
    de gestión también realizan cálculo interactivo (por ejemplo: el procesamiento de transacciones 
    en puntos de ventas).
    \item \textbf{ Software de ingeniería y científico:} está caracterizado por los algoritmos 
    de manejo de números. Las aplicaciones van desde la astronomía a la vulcanología, 
    desde el análisis de la presión de los automotores a la dinámica orbital de las 
    lanzaderas espaciales y desde la biología molecular a la fabricación automática.
    \item \textbf{ Software empotrado:} reside en memoria de sólo lectura y se
    utiliza para controlar productos y sistemas de los mercados industriales y de consumo. El
    software empotrado puede ejecutar funciones muy limitadas y curiosas (por ejemplo: el
    control de las teclas de un horno de microondas) o suministrar una función significativa y
    con capacidad de control (por ejemplo: funciones digitales en un automóvil, tales como
    control de la gasolina, indicadores en el salpicadero, sistemas de frenado, etc.).
    \item \textbf{ Sodtware de computadoras personales:} . El
    procesamiento de textos, las hojas de cálculo, los gráficos por computadora, multimedia,
    entretenimientos, gestión de bases de datos, etc.
    \item \textbf{ Software basado en Web:} Las páginas Web buscadas por un explorador son 
    software que incorpora instrucciones ejecutables y datos.
    \item \textbf{ Software de inteligencia artificial:} hace uso de algoritmos no
    numéricos para resolver problemas complejos para los que no son
    adecuados el cálculo o el análisis directo.
\end{itemize}
\subsubsection{Ciclo de vida del software}
Un marco de referencia que contiene los procesos, las actividades y las
tareas involucradas en el desarrollo, la explotación y el mantenimiento de un
producto de software, abarcando la vida del sistema desde la definición de
los requisitos hasta la finalización de su uso.\\
\subsubsection{Modelos de procesos}
\textbf{¿Qué es un modelo de proceso de software?}\\
Representación formal y simplificada de un proceso de software, presentada desde una 
perspectiva específica.\\
Ejemplos de perspectiva del proceso del hardware.
\begin{itemize}
    \item Flujo de trabajo, secuencia de actividades.
    \item Flujo de datos, flujo de la información.
    \item Rol/acción, quien realiza qué.
\end{itemize}
\textbf{Modelos genéricos}
\begin{enumerate}
    \item \textbf{Lineal secuencial:} sugiere un enfoque sistemático, secuencial, para el
    desarrollo del software. Comprende el análisis, diseño, codificación, pruebas y mantenimiento.
    \item \textbf{Cascada:} Tiene las mismas características que el modelo lineal como se había 
    mencionado anteriormente pero en este se tiene la capacidad de regresar si se detecta un error 
    en cualquiera de las etapas de planeación, desarrollo, diseño, pruebas y mantenimiento del 
    software.
    \item \textbf{Incremental:} El modelo incremental combina elementos del modelo lineal 
    secuencial con la filosofía interactiva de construcción de prototipos. El software se ve como 
    una integración de resultados sucesivos obtenidos después de cada interacción
    \item \textbf{Desarrollo rápido de aplicaciones:} es un modelo de proceso del
    desarrollo del software lineal secuencial que enfatiza un ciclo de desarrollo
    extremadamente corto.\\
    El modelo DRA es una adaptación a alta velocidad del modelo lineal secuencial en el que se 
    logra el desarrollo rápido utilizando una construcción basada en componentes.    
    \item \textbf{Prototipos:}Comienza con la recolección de requisitos. El desarrollador y el 
    cliente encuentran y definen los objetivos globales para el software, identifican los 
    requisitos conocidos y las áreas del esquema en donde es obligatoria más definición.\\
    El prototipo lo evalúa el cliente/usuario y se utiliza para refinar los requisitos del software 
    a desarrollar. La iteración ocurre cuando el prototipo se pone a punto para satisfacer las 
    necesidades del cliente, permitiendo al mismo tiempo que el desarrollador comprenda mejor lo 
    que se necesita hacer.
    \item \textbf{Esprial:}
    \item \textbf{Basados en componentes:} asume que partes del sistema ya existen y se enfoca 
    a su integración.
    
\end{enumerate}

%\subsection{Proceso de gesti\'on de proyecto}
%\subsection{Metodolog\'ias}
%\subsection{Calidad y normas de calidad}
%\subsection{Modelos de madurez}
%\subsection{Temas selectos}

\subsection{}
\end{document}